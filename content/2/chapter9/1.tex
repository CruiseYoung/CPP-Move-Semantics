我们了解了移动语义不能自动传递,所以这对泛型代码会有影响。\par

\hspace*{\fill} \par %插入空行
\textbf{9.1.1 为什么需要完美转发}

要将带有移动语义的对象转发给函数,不仅必须绑定到rvalue引用,而且需要再次使用std::move()将其移动语义转发给另一个函数。\par

例如,引用重载函数的规则解析:\par

\begin{lstlisting}[caption={}]
class X {
	...
};

// forward declarations:
void foo(const X&); // for constant values (read-only access)
void foo(X&); // for variable values (out parameters)
void foo(X&&); // for values that are no longer used (move semantics)
\end{lstlisting}

当调用这些函数时,有以下规则:\par

\begin{lstlisting}[caption={}]
X v;
const X c;

foo(v); // calls foo(X&)
foo(c); // calls foo(const X&)
foo(X{}); // calls foo(X&&)
foo(std::move(v)); // calls foo(X&&)
foo(std::move(c)); // calls foo(const X&)
\end{lstlisting}

假设通过协助函数callFoo()间接地调用相同的参数foo()。函数还需要三个重载:\par

\begin{lstlisting}[caption={}]
void callFoo(const X& arg) { // arg binds to all const objects
	foo(arg); // calls foo(const X&)
}
void callFoo(X& arg) { // arg binds to lvalues
	foo(arg); // calls foo(X&)
}
void callFoo(X&& arg) { // arg binds to rvalues
	foo(std::move(arg)); // needs std::move() to call foo(X&&)
}
\end{lstlisting}

这里,arg都用作左值(具有名称的对象)。第一个版本将其作为const对象转发,但其他两种情况实现了转发非const参数的两种不同方式:\par

\begin{itemize}
	\item 声明为lvalue引用(绑定到没有移动语义的对象)的参数按原样传递。
	\item 声明为rvalue引用(绑定到具有移动语义的对象)的参数通过std::move()传递。
\end{itemize}

这允许我们完美地转发移动语义:对于任何通过移动语义传递的参数,保持移动语义。当遇到没有移动语义的参数时,不添加移动语义。\par

看下callFoo()如何调用不同的foo():\par

\begin{lstlisting}[caption={}]
X v;
const X c;
callFoo(v); // calls foo(X&)
callFoo(c); // calls foo(const X&)
callFoo(X{}); // calls foo(X&&)
callFoo(std::move(v)); // calls foo(X&&)
callFoo(std::move(c)); // calls foo(const X&)
\end{lstlisting}

请记住,传递给rlvalue引用的rvalue在使用时成为lvalue,这意味着需要std::move()再次将其作为rvalue传递。但是,不能在任何地方使用std::move()。对于其他重载,当传递rvalue时,使用std::move()将调用foo()的重载实现来获取rvalue引用。\par

为了在泛型代码中实现完美转发,需要为每个参数进行重载。为了支持所有组合,这意味着对2个泛型参数有9个重载,对3个泛型参数有27个重载。\par

因此,C++11引入了一种特殊的方式来完美地转发给定的参数,而不需要任何重载,但仍然保持类型和具体值。\par

\hspace*{\fill} \par %插入空行
\textbf{完美转发const rvalue引用}

虽然const rvalue引用没有语义上的含义,但如果想要用std::move()标记的常量对象的类型和值,还需要第四个重载:\par

\begin{lstlisting}[caption={}]
void callFoo(const X&& arg) { // arg binds to const rvalues
	foo(std::move(arg)); // needs std::move() to call foo(const X&&)
}
\end{lstlisting}

否则,将调用foo(const X\&)。这通常没问题的,但在某些情况下,可能希望保留传递const rvalue引用的信息(例如,出于任何原因,提供了foo(const X\&\&)重载)。\par

有了完美转发的特性,泛型代码就没必要为了对两个或三个参数,进行16和64次重载了。\par


















