如果想要完美地转发通用Lambda的参数,还必须使用通用引用和std::forward<>()。但现在只需使用auto\&\&来声明通用引用。\par

考虑这个例子,我们已经有了一个函数模板:\par

\begin{lstlisting}[caption={}]
auto callFoo = [](auto&& arg) { // arg is a universal/forwarding reference
	foo(std::forward<decltype(arg)>(arg)); // perfectly forward arg
};

std::string s{"BLM"};
callFoo(s); // OK, arg is std::string&
callFoo(std::move(s)); // OK, arg is std::string&&
\end{lstlisting}

注意,在C++20中,可以通过用模板形参声明Lambda来避免使用decltype(arg)。\par

下面的泛型Lambda使用此函数完美地转发了可变数量的参数:\par

\begin{lstlisting}[caption={}]
[] (auto&&... args) {
	...
	foo(std::forward<decltype(args)>(args)...);
};
\end{lstlisting}

请记住Lambda只是定义函数对象的一种简单方法(定义了operator()的对象允许它们作为函数使用)。上面的定义扩展为编译器定义的类(闭包类型),并将通用引用定义为模板形参:\par

\begin{lstlisting}[caption={}]
class NameDefinedByCompiler {
	...
	public:
	template<typename... Args>
	auto operator() (Args&&... args) const {
		...
		foo(std::forward<decltype(args)>(args)...);
	}
};
\end{lstlisting}

再次注意,用const(或volatile)限定的泛型rvalue引用不是通用引用,这也适用于Lambda。如果形参是用const auto\&\&声明的,只能传递rvalue:\par

\begin{lstlisting}[caption={}]
auto callFoo = [](const auto&& arg) { // arg is not a universal reference
	...
};

const std::string cs{"BLM"};
callFoo(cs); // ERROR: s is not an rvalue
callFoo(std::move(cs)); // OK, arg is const std::string&&
\end{lstlisting}

在Lambda内部,使用std::move()完美地传递传递的参数就足够了。\par
















































