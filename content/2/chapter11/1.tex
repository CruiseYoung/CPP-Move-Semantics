我们经常需要将返回值传递给另一个函数:\par

\begin{lstlisting}[caption={}]
// pass return value of compute() to process():
process(compute(t)); // OK, uses perfect forwarding of returned value
\end{lstlisting}

非泛型代码中,需要知道所涉及的类型。然而,在泛型代码中,也希望compute()的返回值完全传递给process()。\par

好消息是:如果直接将返回值传递给另一个函数,该值会完美传递,保持其类型和值类别。不必担心移动语义,如果支持将自动使用。\par

\hspace*{\fill} \par %插入空行
\textbf{11.1.1 默认完美传递的细节}

这是一个完整的例子:\par

{\color{red}{generic/perfectpassing.cpp}}\par

\begin{lstlisting}[caption={}]
#include <iostream>
#include <string>

void process(const std::string&) {
	std::cout << "process(const std::string&)\n";
}
void process(std::string&) {
	std::cout << "process(std::string&)\n";
}
void process(std::string&&) {
	std::cout << "process(std::string&&)\n";
}

const std::string& computeConstLRef(const std::string& str) {
	return str;
}
	std::string& computeLRef(std::string& str) {
	return str;
}
	std::string&& computeRRef(std::string&& str) {
	return std::move(str);
}
	std::string computeValue(const std::string& str) {
	return str;
}

int main()
{
	process(computeConstLRef("tmp")); // calls process(const std::string&)
	
	std::string str{"lvalue"};
	process(computeLRef(str)); // calls process(std::string&)
	
	process(computeRRef("tmp")); // calls process(std::string&&)
	process(computeRRef(std::move(str))); // calls process(std::string&&)
	
	process(computeValue("tmp")); // calls process(std::string&&)
}
\end{lstlisting}

\begin{itemize}
	\item 如果compute()返回一个const lvalue引用:\par
	\begin{lstlisting}[caption={}]
	const std::string& computeConstLRef(const std::string& str) {
		return str;
	}
	\end{lstlisting}
	返回值的值类别是一个lvalue,这意味着返回值会完美转发,并与const lvalue引用匹配:\par
	\begin{lstlisting}[caption={}]
	process(computeConstLRef("tmp")); // calls process(const std::string&)
	\end{lstlisting}
	\item 如果compute()返回一个非const lvalue引用:\par
	\begin{lstlisting}[caption={}]
	std::string& computeLRef(std::string& str) {
		return str;
	}
	\end{lstlisting}
	返回值的值类别是一个;value,这意味着返回值会完全转发,并与非const lvalue引用的最佳匹配:\par
	\begin{lstlisting}[caption={}]
	std::string str{"lvalue"};
	process(computeLRef(str)); // calls process(std::string&)
	\end{lstlisting}
	\item 如果compute()返回一个rvalue引用:\par
	\begin{lstlisting}[caption={}]
	std::string&& computeRRef(std::string&& str) {
		return std::move(str);
	}
	\end{lstlisting}
	返回值的值类别是xvalue,这意味着返回值会完全转发为rvalue引用,允许process()窃取它的值:\par
	\begin{lstlisting}[caption={}]
	process(computeRRef("tmp")); // calls process(std::string&&)
	process(computeRRef(std::move(str))); // calls process(std::string&&)
	\end{lstlisting}
	\item 如果compute()按值返回一个临时对象:\par
	\begin{lstlisting}[caption={}]
	std::string computeValue(const std::string& str) {
		return str;
	}
	\end{lstlisting}
	返回值的值类别是prvalue,这意味着返回值被完全转发为rvalue引用,也允许process()偷取它的值:\par
	\begin{lstlisting}[caption={}]
	process(computeValue("tmp")); // calls process(std::string&&)
	\end{lstlisting}
\end{itemize}

注意,通过返回const值:\par

\begin{lstlisting}[caption={}]
const std::string computeConstValue(const std::string& str) {
	return str;
}
\end{lstlisting}

或const rvalue引用:\par

\begin{lstlisting}[caption={}]
const std::string&& computeConstRRef(std::string&& str) {
	return std::move(str);
}
\end{lstlisting}

再次禁用了移动语义:\par

\begin{lstlisting}[caption={}]
process(computeConstValue("tmp")); // calls process(const std::string&)
process(computeConstRRef("tmp")); // calls process(const std::string&)
\end{lstlisting}

如果有const\&\&的声明,可以接受这样的重载。\par

因此:不要将value返回的值标记为const,也不要将返回的非const rvalue引用标记为const。\par


