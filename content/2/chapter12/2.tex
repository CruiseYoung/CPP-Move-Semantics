就像其他占位符类型auto一样,decltype(auto)是一个占位符类型,它让编译器在初始化时推断出类型。然而,在这种情况下,该类型是根据decltype的规则推导出来的:\par

\begin{itemize}
	\item 如果用普通名称初始化它或返回普通名称,则返回类型是具有该名称的对象的类型。
	\item 如果使用表达式初始化或返回表达式,则返回类型为求值表达式的类型和值类别:
	\begin{itemize}
		\item[-] 对于\textbf{prvalue},它只产生它的值类型:type
		\item[-] 对于\textbf{lvalue},它将其类型作为lvalue引用:type\&
		\item[-] 对于\textbf{xvalue},它将其类型作为rvalue引用:type\&\&
	\end{itemize}
\end{itemize}

例如:\par

\begin{lstlisting}[caption={}]
std::string s = "hello";
std::string& r = s;

// initialized with name:
decltype(auto) da1 = s; // std::string
decltype(auto) da2(s); // same
decltype(auto) da3{s}; // same
decltype(auto) da4 = r; // std::string&

// initialized with expression:
decltype(auto) da5 = std::move(s); // std::string&&
decltype(auto) da6 = s+s; // std::string
decltype(auto) da7 = s[0]; // char&
decltype(auto) da8 = (s); // std::string&
\end{lstlisting}

对于表达式,根据规则,类型推导如下:\par

\begin{itemize}
	\item 因为std::move(s)是一个xvalue,所以da5是一个rvalue引用。
	\item 因为字符串的加法操作符按值返回一个新的临时字符串(因此它是一个prvalue),所以da6是一个普通值类型。
	\item 因为s[0]返回对第一个字符的lvalue引用,所以它是一个lvalue,并强制da7也是一个lvalue引用。
	\item 因为(s)是一个lvalue,所以da8是一个lvalue引用。是的,这里的括号会有所不同。
\end{itemize}

与总是引用的auto\&\&相反,decltype(auto)有时只是一个值(如果用值类型对象的名称或用prvalue表达式初始化)。\par

注意decltype(auto)不能有其他限定符:\par

\begin{lstlisting}[caption={}]
	decltype(auto) da{s}; // OK
	const decltype(auto)& da1{s}; // ERROR
	decltype(auto)* da2{&s}; // ERROR
\end{lstlisting}

\hspace*{\fill} \par %插入空行
\textbf{12.2.1 返回类型的decltype(auto)}

当使用decltype(auto)作为返回类型时,使用decltype的规则如下:\par

\begin{itemize}
	\item 如果表达式返回/产生一个普通值,那么值类别是一个prvalue, decltype(auto)推导出一个值类型。
	\item 如果表达式返回/产生一个lvalue引用,那么值类别是一个lvalue,decltype(auto)推导出一个lvalue引用。
	\item 如果表达式返回/产生一个rvalue引用,那么值类别是一个xvalue, decltype(auto)推导出一个rvalue引用。
\end{itemize}

这正是完美返回所需要的:对于普通值,我们推导一个值;对于引用,我们推导一个相同类型的引用。\par

一个更通用的例子,考虑一个框架的helper函数(在一些初始化或日志记录之后)透明地调用一个函数,就像直接调用它一样:\par

{\color{red}{generic/call.hpp}}

\begin{lstlisting}[caption={}]
	#include <utility> // for forward<>()
	template <typename Func, typename... Args>
	decltype(auto) call (Func f, Args&&... args)
	{
		...
		return f(std::forward<Args>(args)...);
	}
\end{lstlisting}

该函数将args声明为一个可变数量的通用引用(也称为转发引用)。通过std::forward<>(),它完美地将这些给定的参数转发给f作为第一个参数传递的函数。因为我们使用decltype(auto)作为返回类型,所以完美地将f()的返回值返回给call()的调用者。因此,可以同时调用按值返回的函数和按引用返回的函数。例如:\par

{\color{red}{generic/call.cpp}}

\begin{lstlisting}[caption={}]
#include "call.hpp"
#include <iostream>
#include <string>

std::string nextString()
{
	return "Let's dance";
}

std::ostream& print(std::ostream& strm, const std::string& val)
{
	return strm << "value: " << val;
}

std::string&& returnArg(std::string&& arg)
{
	return std::move(arg);
}

int main()
{
	auto s = call(nextString); // call() returns temporary object
	
	auto&& ref = call(returnArg, std::move(s)); // call() returns rvalue reference to s
	std::cout << "s: " << s << '\n';
	std::cout << "ref: " << ref << '\n';
	
	auto str = std::move(ref); // move value from s and ref to str
	std::cout << "s: " << s << '\n';
	std::cout << "ref: " << ref << '\n';
	std::cout << "str: " << str << '\n';
	
	call(print, std::cout, str) << '\n'; // call() returns reference to std::cout
}
\end{lstlisting}

当调用\par

\begin{lstlisting}[caption={}]
	auto s = call(nextString);
\end{lstlisting}

函数call()调用函数nextString(),不带任何参数,并返回它的返回值来初始化s。\par

当调用\par

\begin{lstlisting}[caption={}]
	auto&& ref = call(returnArg, std::move(s));
\end{lstlisting}

函数调用()调用带有std::move()标记的函数returnArg()。returnArg()将传递的参数作为右值引用返回,然后call()完美地返回给调用者来初始化ref。str仍然有它的值并且ref会引用它:\par

\begin{tcolorbox}[colback=white,colframe=black]
s: Let's dance \\
ref: Let's dance
\end{tcolorbox}

使用\par

\begin{lstlisting}[caption={}]
auto str = std::move(ref);
\end{lstlisting}

将s和ref的值都移动到str,得到以下状态:\par

\begin{tcolorbox}[colback=white,colframe=black]
s: \\
ref: \\
str: Let's dance
\end{tcolorbox}

当调用\par

\begin{lstlisting}[caption={}]
call(print, std::cout, ref) << '\n';
\end{lstlisting}

函数调用()使用std::cout和ref作为完全转发的参数调用print()函数。Print()将传递的流作为lvalue引用返回,然后完美地返回给call()的调用者。\par

\hspace*{\fill} \par %插入空行
\textbf{12.2.2 延迟完美返回}

为了完美地返回之前计算的值,我们必须使用decltype(auto)声明一个局部对象,当且仅当它是一个rvalue引用时,使用std::move()返回它。例如:\par

\begin{lstlisting}[caption={}]
	template<typename Func, typename... Args>
	decltype(auto) call(Func f, Args&&... args)
	{
		decltype(auto) ret{f(std::forward<Args>(args)...)};
		...
		if constexpr (std::is_rvalue_reference_v<decltype(ret)>) {
			return std::move(ret); // move xvalue returned by f() to the caller
		}
		else {
			return ret; // return the plain value or the lvalue reference
		}
	}
\end{lstlisting}

这个函数中,ret的类型就是f()的完美推导类型。通过使用if constexpr(从C++17起),可以使用decltype(auto)和decltype两种方式来推导一个类型,如下所示:\par

\begin{itemize}
	\item 如果ret声明为rvalue引用,decltype(auto)使用表达式std::move(ret),这是一个xvalue,来推导rvalue引用。因此,将f()返回的值移动到这个函数的调用者。
	\item 如果ret声明为普通值或lvalue引用,decltype(auto)使用名为ret的类型,这也是一个值类型或lvalue引用类型。
\end{itemize}

其他的解决方案并不总是有效:\par

\begin{itemize}
	\item 即使在C++20之前,以下内容也会做正确的事情,但有性能问题:\par
	\begin{lstlisting}[caption={}]
		decltype(auto) call( ... )
		{
			decltype(auto) ret{f( ... )};
			...
			return static_cast<decltype(ret)>(ret); // perfect return but unnecessary copy
		}
	\end{lstlisting}
	事实上,我们总是使用static\_cast<>可能会禁用移动语义和复制备选。对于普通值,这就像在return语句中有一个不必要的std::move()。
	\item 简单地返回ret并不总是有效的:
	\begin{lstlisting}[caption={}]
		decltype(auto) call( ... )
		{
			decltype(auto) ret{f( ... )};
			...
			return ret; // may be an ERROR
		}
	\end{lstlisting}
	call()的返回类型是正确的。但是,如果f()返回rvalue引用,则不能返回左值ret,因为非const重值引用没有绑定到lvalue。
	\item 使用auto\&\&来声明ret不起作用,因为你将通过引用返回:
	\begin{lstlisting}[caption={}]
		decltype(auto) call( ... )
		{
			auto&& ret{f( ... )};
			...
			return ret; // fatal runtime error: returns a reference to a local object
		}
	\end{lstlisting}
\end{itemize}

使用decltype(auto)时,不要在返回的名字后面加括号:\par

\begin{lstlisting}[caption={}]
	decltype(auto) call( ... )
	{
		decltype(auto) ret{f( ... )};
		...
		if constexpr (std::is_rvalue_reference_v<decltype(ret)>) {
			return std::move(ret); // move value returned by f() to the caller
		}
		else {
			return (ret); // FATAL RUNTIME ERROR: always returns an lvalue reference
		}
	}
\end{lstlisting}

这样,返回类型decltype(auto)会切换到表达式规则,并总是推断出一个lvalue引用,因为ret是一个lvalue(一个有名称的对象)。\par

如果习惯在return语句中把名字和表达式用括号括起来,那就不要这样做了。若继续括起来,使用decltype(auto)时可能会出现错误。\par

\hspace*{\fill} \par %插入空行
\textbf{12.2.3 完美的Lambda转发和返回}

如果Lambda完美返回,则必须更改它的返回类型。声明如下:\par

\begin{lstlisting}[caption={}]
	[] (auto f, auto&&... args) {
		...
	}
\end{lstlisting}

代表:\par

\begin{lstlisting}[caption={}]
	[] (auto f, auto&&... args) -> auto {
		...
	}
\end{lstlisting}

这意味着在默认情况下,Lambda总是按值返回。\par

通过用返回类型decltype(auto)显式声明Lambda,可以实现完美返回:\par

\begin{lstlisting}[caption={}]
	[] (auto f, auto&&... args) -> decltype(auto) {
		...
		return f(std::forward<decltype(args)>(args)...);
	};
\end{lstlisting}

对于延迟完美返回,我们需要和前面介绍的一样的技巧:如果返回一个rvalue引用,则必须使用std::move():\par

\begin{lstlisting}[caption={}]
	[] (auto f, auto&&... args) -> decltype(auto) {
		decltype(auto) ret = f(std::forward<decltype(args)>(args)...);
		...
		if constexpr (std::is_rvalue_reference_v<decltype(ret)>) {
			return std::move(ret); // move value returned by f() to the caller
		}
		else {
			return ret; // return the value or the lvalue reference
		}
	};
\end{lstlisting}

同样,不要在返回的名称周围加上额外的括号,因为decltype(auto)总是推断为lvalue引用:\par

\begin{lstlisting}[caption={}]
	[] (auto f, auto&&... args) -> decltype(auto) {
		...
		decltype(auto) ret = f(std::forward<decltype(args)>(args)...);
		...
		if constexpr (std::is_rvalue_reference_v<decltype(ret)>) {
			return std::move(ret); // move value returned by f() to the caller
		}
		else {
			return (ret); // FATAL RUNTIME ERROR: always returns an lvalue reference
		}
	};
\end{lstlisting}










