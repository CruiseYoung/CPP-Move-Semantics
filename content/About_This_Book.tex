\begin{flushright}
	\zihao{0} 关于本书
\end{flushright}

这本书详解C++的移动语义。从基本原理出发,解释了移动语义的所有特性和案例,这样就可以正确地理解和使用移动语义。\par

本书重点在于所描述的特性,需要在实践中进行应用。示例和背景信息,有助于理解和改进简单的类型,甚至是泛型库和框架的代码。\par

\hspace*{\fill} \par %插入空行
\textbf{阅前须知}

需要您熟悉C++,应该熟悉类和引用的一般概念,并且应该能使用C++标准库的iostream和容器等组件来编写C++程序。还应该熟悉“现代C++”的基本特性,比如:atuo或基于范围的for循环。\par

我的目标是让普通C++程序员能够理解这些内容,他们不一定知道最新特性的所有细节。我将讨论基本特性,并在需要时回顾更细节的问题。\par

这也确保了专家和中级程序员都也可以阅读本书。\par


\hspace*{\fill} \par %插入空行
\textbf{本书结构}

这本书涵盖了C++的移动语义,以及C++20的各个方面。这既适用于语言和库特性,也适用于日常应用程序编程的特性和(基础)库复杂实现的特性。\par

不同的章节是分组的,所以应该从头到尾的阅读。后面的章节通常依赖于前面章节介绍的特性,但交叉引用在特定的后续主题中也有帮助,并且会指出在何处引用前面介绍的特性。\par

本书包含以下部分:\par
\begin{itemize}
	\item \textbf{Part I}介绍了移动语义的基本特性(特别是非泛型代码)。
	\item \textbf{Part II}介绍了泛型代码(特别是在模板和泛型Lambda中使用的)的移动语义的特性。
	\item \textbf{Part III}包含在C++标准库中移动语义的使用(也给出了一个在实践中如何使用移动语义的例子)。
\end{itemize}


\hspace*{\fill} \par %插入空行
\textbf{如何阅读}

不要畏惧这本书的页数。当你研究细节(比如实现模板)时,事情会变得非常复杂。对于基本的理解,本书的前三分之一(第一部分,特别是第1-5章)就足够了。\par

以我的经验,学习新东西的最好方法是看例子。因此,你会在书中找到很多例子。有些只是说明抽象概念的几行代码,有些会提供具体应用的完整程序。后一种示例将通过描述程序代码的文件的C++注释来介绍。你可以在这本书的网站http://www.cppmove.com上找到这些文件。\par

\hspace*{\fill} \par %插入空行
\textbf{实现方式}

关于我编写代码和注释的方式,请注意以下提示。\par

\hspace*{\fill} \par %插入空行
\textbf{初始化}

我通常使用带大括号的现代初始化形式(C++11中引入的统一初始化):\par
\begin{lstlisting}[caption={}]
int i{42};
std::string s{"hello"};
\end{lstlisting}

这种形式的初始化称为大括号初始化,有以下优点:\par
\begin{itemize}
	\item 可以与基本类型、类类型、聚合、枚举类型和auto一起使用
	\item 可以用于初始化带有多个值的容器
	\item 可以检测截断错误(例如,用浮点值初始化int)
	\item 不能与函数声明或调用混淆
\end{itemize}

如果大括号为空,则调用(子)对象的默认构造函数,并保证使用0/false/nullptr初始化基本数据类型。\par

\hspace*{\fill} \par %插入空行
\textbf{错误描述}

我会经常谈论编程错误。如果没有特殊提示,错误或注释为:\par

\begin{lstlisting}[caption={}]
... // ERROR
\end{lstlisting}

\hspace*{\fill} \par %插入空行
表示编译时错误。对应的代码不应该编译(使用符合标准的编译器)。\par

如果使用运行时错误,程序会编译通过,但不能正确运行或导致未定义的行为发生(因此,可能会执行预期的操作)。\par

\hspace*{\fill} \par %插入空行
\textbf{代码简化}

我试图用有用的例子来解释所有的特征。然而,为了专注于关键方面,我可能经常会跳过一部分代码。\par

\begin{itemize}
	\item 大多数时候,使用省略号(“…”)来表示其他代码。注意,这里没有使用代码。如果看到带有代码字体的省略号,那么代码必须有这三个点作为语言特性(例如“typename…”)。
	\item 头文件中,我通常跳过预处理器保护。所有头文件应该有类似如下的内容:
	\begin{lstlisting}[caption={}]
	#ifndef MYFILE_HPP
	#define MYFILE_HPP
	...
	#endif // MYFILE_HPP
	\end{lstlisting}
	因此,在项目中使用这些头文件时,请补全代码。
\end{itemize}

\hspace*{\fill} \par %插入空行
\textbf{C++标准}

不同的C++标准定义了不同的C++版本。\par

最初的C++标准出版于1998年,随后在2003年通过勘误表进行了修订,对原来的标准进行了微小的修正。“旧C++标准”即为C++98或C++03标准。\par

“现代C++”始于C++11,并在C++14和C++17中扩展。国际C++标准委员会现在的目标是每三年发布一个新标准。显然,这就减少了进行大规模添加的时间,可以更快地为编程社区带来变化。\par

因此,开发更大的特性需要时间,而且可能涉及多个标准。下一个“更现代的C++”已经在地平线上,如C++20所介绍的那样。同样,编程的方式可能会改变,但编译器需要一些时间来提供最新的语言特性。在撰写本书时,C++17是主流编译器支持的最新标准。\par

幸运的是,在C++11和C++14中都介绍了移动语义的基本原理。因此,本书中的代码示例,通常可以在主流编译器的最新版本上编译。如果讨论C++17或C++20引入的特殊特性,我会进行说明。\par

\hspace*{\fill} \par %插入空行
\textbf{示例代码的信息}

你可以访问所有的示例程序,并从它的网站上找到更多关于这本书的信息,它的网址如下:\\http://www.cppmove.com\par

\hspace*{\fill} \par %插入空行
\textbf{反馈}

欢迎消极的和积极的建设性意见,希望你会发现这是一本优秀的书。然而,有时我停止写作、审查和调整,以便“发布新版本”。因此,你会发现错误、不一致、可以改进的示例或缺失的主题。你们的反馈给了我解决这些问题的机会,我会通过这本书的网站告知所有读者这些变化,并改进后续的修订或版本。\par

联系我最好的方式是通过电子邮件。你可以在这本书的网站上找到电子邮件地址:\\http://www.cppmove.com\par

如果你使用电子书,你可能想要确保有这本书的最新版本(记住它是逐步编写和出版的)。在提交报告之前,您还应该检查该书的Web站点以获取当前已知的勘误表。无论如何,在反馈意见时,请参考本版本的发布日期。目前出版日期为2020年12月19日(也可以在第二页,封面后的下一页找到)。\par

非常感谢!\par

\newpage




















