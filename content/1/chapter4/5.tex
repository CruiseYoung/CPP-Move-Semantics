\begin{itemize}
	\item 避免有名称的对象。
	\item 避免不必要的\textit{std::move()}。尤其不要在返回局部对象时使用。
	\item 从形参初始化成员的构造函数(对于这种构造函数,移动操作很廉价),应该按值接受实参并将其移动到成员。
	\item 从形参初始化成员的构造函数,移动操作需要大量时间,应该重载移动语义以获得最佳性能。
	\item 一般来说,从参数中创建和初始化新值(对于移动操作来说成本很低)应该按值和移动的方式接受参数。但是,不要以值为单位来更新/修改现有的值。
	\item 不要在派生类中声明虚析构函数(除非必须实现)。
\end{itemize}


\newpage