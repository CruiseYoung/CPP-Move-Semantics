是否自动生成以及自动生成哪个特殊成员函数取决于刚才描述的规则,许多程序员并不知道这些规则。因此,即使在C++11之前,指导方针是不提供任何或所有用于复制、赋值和销毁的特殊成员函数。\par

\begin{itemize}
	\item C++11之前,这条原则被称为“3法则”:要么声明全部三种(复制构造函数、赋值操作符和析构函数),要么一个都不声明。
	\item 从C++11开始,该规则就变成了“5规则”,通常的表述方式是:要么声明所有5种(复制构造函数、移动构造函数、复制赋值操作符、移动赋值操作符和析构函数),要么一个都不声明。
\end{itemize}

在这里,声明的意思是:\par

\begin{itemize}
	\item 要么实现 (\{...\})
	\item 或者声明为默认 (=default)
	\item 或声明为已删除 (=delete)
\end{itemize}

当其中一个特殊成员函数被实现、默认或删除时,应该实现、默认或删除所有其他四个特殊成员函数。\par

但是,您应该注意这条规则。我建议您更多地把它作为一个指导方针,当其中一个特殊成员函数是用户声明的时候,仔细考虑所有这5个特殊成员函数。\par

为了只启用复制语义,你应该=默认复制特殊成员函数,而不声明特殊的移动成员函数(删除和默认特殊移动成员函数是行不通的,实现它们会使类变得复杂)。如果生成的移动语义创建了无效状态,则建议使用此选项,我们将在关于无效已移动状态的章节中讨论这一点。\par

当应用5规则时,有时程序员使用它来为新的移动操作添加声明,却不理解这意味着什么。程序员只是用=default来声明移动操作,因为已经实现了复制操作,他们希望遵循5规则。\par

因此,我通常教授“5法则”或“3法则”:\par

\begin{itemize}
	\item 如果声明了复制构造函数、移动构造函数、复制赋值操作符、移动赋值操作符或析构函数,则必须仔细考虑如何处理其他特殊成员函数。
	\item 如果不理解移动语义,只考虑复制构造函数、复制赋值操作符和析构函数(如果声明其中之一)。如果有疑问,可以使用=default声明复制特殊成员函数来禁用移动语义。
\end{itemize}






















































