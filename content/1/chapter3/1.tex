假设有一个相当简单的类,其中的成员类型的移动语义可以让其不一样:\par

{\color{red}{basics/customer.hpp}}

\begin{lstlisting}[caption={}]
#include <string>
#include <vector>
#include <iostream>
#include <cassert>

class Customer {
private:
	std::string name; // name of the customer
	std::vector<int> values; // some values of the customer
public:
	Customer(const std::string& n)
	: name{n} {
		assert(!name.empty());
	}

	std::string getName() const {
		return name;
	}

	void addValue(int val) {
		values.push_back(val);
	}

	friend std::ostream& operator<< (std::ostream& strm, const Customer& cust) {
		strm << '[' << cust.name << ": ";
		for (int val : cust.values) {
			strm << val << ' ';
		}
		strm << ']';
		return strm;
	}
};
\end{lstlisting}

这个类有两个(可能)开销很大的成员,一个是字符串name,一个是整型vector:\par

\begin{lstlisting}[caption={}]
class Customer {
private:
	std::string name; // name of the customer
	std::vector<int> values; // some values of the customer
	...
};
\end{lstlisting}

复制这两个成员的成本很高\par

\begin{itemize}
	\item 要复制name,我们必须为字符串的字符分配内存(除非name很短,并且使用小字符串优化(SSO)实现字符串)。
	\item 要复制这些值,必须为vector的元素分配内存。
\end{itemize}

如果有一个string类型的vector或者其他大型元素类型,那么开销会更大。例如,string类型vector的副本必须为元素的动态数组和每个元素所需的内存分配内存。\par

好消息是,这样的类通常自动支持移动语义。自C++11以来,编译器通常会生成移动构造函数和移动赋值操作符(类似于自动生成复制构造函数和复制赋值操作符)。\par

这样会有以下效果:\par

\begin{itemize}
	\item 按值返回本地Customer将使用移动语义(如果它没有优化掉的话)。
	\item 通过值传递一个未命名的Customer对象将使用移动语义(如果没有优化掉的话)。
	\item 按值传递一个临时的Customer对象(例如,由另一个函数返回)将使用移动语义(如果它没有优化掉的话)。
	\item 通过值传递一个标有\textit{std::move()}的Customer对象将使用移动语义(如果它没有被优化掉的话)。
\end{itemize}

例如:\par

{\color{red}{basics/customer1.cpp}}

\begin{lstlisting}[caption={}]
#include "customer.hpp"
#include <iostream>
#include <random>

#include <utility> // for std::move()

int main()
{
	// create a customer with some initial values:
	Customer c{"Wolfgang Amadeus Mozart" };
	for (int val : {0, 8, 15}) {
		c.addValue(val);
	}
	std::cout << "c: " << c << '\n'; // print value of initialized c
	
	// insert the customer twice into a collection of customers:
	std::vector<Customer> customers;
	customers.push_back(c); // copy into the vector
	customers.push_back(std::move(c)); // move into the vector
	std::cout << "c: " << c << '\n'; // print value of moved-from c
	
	// print all customers in the collection:
	std::cout << "customers:\n";
	for (const Customer& cust : customers) {
		std::cout << " " << cust << '\n';
	}
}
\end{lstlisting}

这里,我们创建并初始化一个customer c(为了避免SSO,我们使用一个相当长的名称)。初始化c后,第一个输出如下:\par

\begin{tcolorbox}[colback=white,colframe=black]
c: [Wolfgang Amadeus Mozart: 0 8 15 ]
\end{tcolorbox}

然后将这个customer插入到vector中两次:复制一次,移动一次:\par

\begin{lstlisting}[caption={}]
customers.push_back(c); // copy into the vector
customers.push_back(std::move(c)); // move into the vector
\end{lstlisting}

然后,c的输出:\par

\begin{tcolorbox}[colback=white,colframe=black]
c: [: ]
\end{tcolorbox}

第二次调用push\_back()时,名称和值都移到了vector的第二个元素中。但是,已移动的对象处于有效但未指定的状态。因此,第二个输出可以有任何值name和values:\par

\begin{itemize}
	\item 可能仍然有相同的值:\par
	\begin{tcolorbox}[colback=white,colframe=black]
	c: [Wolfgang Amadeus Mozart: 0 8 15 ]
	\end{tcolorbox}
	\item 可能有完全不同的值:\par
	\begin{tcolorbox}[colback=white,colframe=black]
	c: [value was moved away: 0 ]
	\end{tcolorbox}
\end{itemize}

然而,移动语义是为了优化性能而提供,而分配不同的值并不一定是提高性能,所以在实现中通常把字符串和vector都设为空。\par

在任何情况下,我们都可以看到为Customer类自动启用了移动语义。出于同样的原因,现在可以保证以下代码是高效的:\par

\begin{lstlisting}[caption={}]
Customer createCustomer()
{
	Customer c{ ... };
	...
	return c; // uses move semantics if not optimized away
}

std::vector<Customer> customers;
...
customers.push_back(createCustomer()); // uses move semantics
\end{lstlisting}

详见basics/customer2.cpp的示例。\par

重要的是,自从C++11以来,类中能使用移动语义的成员,就会使用移动语义。这些类有:\par

\begin{itemize}
	\item 如果从不需要的源创建新对象,则使用成员的移动构造函数:\par
	\begin{lstlisting}[caption={}]
	Customer c1{ ... }
	...
	Customer c2{std::move(c1)}; // move members of c1 to members of c2
	\end{lstlisting}
	\item 移动赋值操作符如果从一个不再需要该的源赋值。\par
	\begin{lstlisting}[caption={}]
	Customer c1{ ... }, c2{ ... };
	...
	c2 = std::move(c1); // move assign members of c1 to members of c2
	\end{lstlisting}
\end{itemize}

注意,通过显式实现以下改进,这样的类可以从移动语义获益更多:\par

\begin{itemize}
	\item 初始化成员时使用移动语义
	\item 使用移动语义使获取函数既安全又快速
\end{itemize}

\hspace*{\fill} \par %插入空行
\textbf{3.1.1 什么时候使用自启动移动语义的类?}

编译器可以自动生成特殊的移动成员函数(移动构造函数和移动赋值操作符)。然而,也有一些限制。约束是编译器必须假定生成的操作是正确的。正确的做法是优化正常的复制行为:我们移动成员,而不是复制成员。\par

如果类改变了复制或赋值的常规行为,那么在优化这些操作时可能也必须做一些不同的事情。因此,当以下至少一个特殊成员函数被用户声明时,将禁用移动操作的自动生成:\par

\begin{itemize}
	\item 拷贝构造函数
	\item 拷贝赋值运算符
	\item 另一个移动操作
	\item 析构函数
\end{itemize}

注意,是“用户声明”。任何形式的复制构造函数、复制赋值操作符或析构函数的显式声明都禁用移动语义。例如,如果我们实现了一个什么也不做的析构函数,就禁用了移动语义:\par

\begin{lstlisting}[caption={}]
class Customer {
	...
	~Customer() { // automatic move semantics is disabled
	}
};
\end{lstlisting}

即使下面的声明也足以禁用移动语义:\par

\begin{lstlisting}[caption={}]
class Customer {
	...
	~Customer() = default; // automatic move semantics is disabled
};
\end{lstlisting}

用户声明析构函数的行为为默认的,因此禁用了移动语义。通常,这种情况下,复制语义将作为一种退阶选择。\par

因此没有特定的需要,就不要实现或声明析构函数(很多程序员都没有遵循这一规则)。\par

这也意味着默认情况下,多态基类禁用了移动语义:\par

\begin{lstlisting}[caption={}]
class Base {
	...
	virtual ~Base() { // automatic move semantics is disabled
	}
};
\end{lstlisting}

注意,这意味着移动语义只对在这个基类中声明的成员禁用。对于派生类的成员,移动语义仍然会自动生成(如果派生类没有显式声明特殊的成员函数)。在套路类层次中的移动语义时会讨论了这一点。\par

\hspace*{\fill} \par %插入空行
\textbf{3.1.2 生成的移动操作有负面影响的情况}

请注意,生成的移动操作可能会引入问题,即使生成的复制操作工作正确。特别是在以下情况,必须小心:\par

\begin{itemize}
	\item 对成员变量有限制
	\begin{itemize}
		\item[-] 值有限制
		\item[-] 值相互依赖
	\end{itemize}
	\item 使用了引用语义的成员(指针,智能指针,…)
	\item 对象没有默认构造
\end{itemize}

可能出现的问题是,已移动的对象可能不再有效:可能破坏不变量,或者对象的析构函数失败。例如,本章中的Customer类的对象可能突然有一个空名称,即使我们有断言来避免这种情况。关于已移动的那一章将对此进行详细讨论。\par











