引用限定符允许对特定类别的对象调用函数时,以不同的方式实现函数。目标是在为不再需要其值的对象,调用不同的成员函数。\par

虽然确实有这个特性,但没有使用。可以(也应该)用它来确保修改对象的操作,不会让即将销毁的临时对象调用。\par

\hspace*{\fill} \par %插入空行
\textbf{5.3.1 赋值操作符的引用限定符}

更好地使用引用限定符的方式是修改赋值操作符的实现。如http://wg21.link/n2819中建议的那样,在可能的地方使用引用限定符声明赋值操作符可能会更好。\par

例如,字符串的赋值操作符声明如下:\par

\begin{lstlisting}[caption={}]
namespace std {
	template<typename charT, ... >
	class basic_string {
	public:
		...
		constexpr basic_string& operator=(const basic_string& str);
		constexpr basic_string& operator=(basic_string&& str) noexcept( ... );
		constexpr basic_string& operator=(const charT* s);
		...
	};
}
\end{lstlisting}

这允许将新值赋给临时字符串:\par

\begin{lstlisting}[caption={}]
std::string getString();

getString() = "hello"; // OK
foo(getString() = ""); // passes string instead of bool
\end{lstlisting}

考虑用引用限定符声明赋值操作符:\par

\begin{lstlisting}[caption={}]
namespace std {
	template<typename charT, ... >
	class basic_string {
	public:
		...
		constexpr basic_string& operator=(const basic_string& str) &;
		constexpr basic_string& operator=(basic_string&& str) & noexcept( ... );
		constexpr basic_string& operator=(const charT* s) &;
		...
	};
}
\end{lstlisting}

代码将不再编译:\par

\begin{lstlisting}[caption={}]
std::string getString();

getString() = "hello"; // ERROR
foo(getString() = ""); // ERROR
\end{lstlisting}

注意,特别是对于可以用作布尔值的类型,将有助于找到如下错误:\par

\begin{lstlisting}[caption={}]
std::optional<int> getValue();

if (getValue() = 0) { // OOPS: compiles although = is used instead of ==
	...
}
\end{lstlisting}

其实,会给临时对象返回基本类型的属性:右值。\par

注意,所有这些修改C++标准的建议都被拒绝了,主要是考虑向后兼容性。然而,在实现自己的类时,可以使用以下改进的方式:\par

\begin{lstlisting}[caption={}]
class MyType {
public:
	...
	// disable assigning value to temporary objects:
	MyType& operator=(const MyType& str) & =default;
	MyType& operator=(MyType&& str) & =default;
	
	// because this disables the copy/move constructor, also:
	MyType(const MyType&) =default;
	MyType(MyType&&) =default;
	...
};
\end{lstlisting}

通常,对可能修改对象的成员函数执行此操作。\par

\hspace*{\fill} \par %插入空行
\textbf{5.3.2 其他成员函数的引用限定符}

如getter示例所示,引用限定符也可以而且应该在返回对对象的引用时使用。这样,可以减少访问已销毁临时对象的成员的风险。\par

同样,标准字符串的当前声明可以作为一个例子:\par

\begin{lstlisting}[caption={}]
namespace std {
	template<typename charT, ... >
	class basic_string {
	public:
		...
		constexpr const charT& operator[](size_type pos) const;
		constexpr charT& operator[](size_type pos);
		constexpr const charT& at(size_type n) const;
		constexpr charT& at(size_type n);
		
		constexpr const charT& front() const;
		constexpr charT& front();
		constexpr const charT& back() const;
		constexpr charT& back();
		...
	};
}
\end{lstlisting}

相反,下面的重载会更好一些:\par

\begin{lstlisting}[caption={}]
namespace std {
	template<typename charT, ... >
	class basic_string {
	public:
		...
		constexpr const charT& operator[](size_type pos) const&;
		constexpr charT& operator[](size_type pos) &;
		constexpr charT operator[](size_type pos) &&;
		constexpr const charT& at(size_type n) const&;
		constexpr charT& at(size_type n) &;
		constexpr charT at(size_type n) &&;
		
		constexpr const charT& front() const&;
		constexpr charT& front() &;
		constexpr charT front() &&;
		constexpr const charT& back() const&;
		constexpr charT& back() &;
		constexpr charT back() &&;
		...
	};
}
\end{lstlisting}

同样,由于向后兼容性,C++标准中相应的更改可能会成为一个问题。但可以为自定义的类型提供这些重载。这种情况下,不要忘记右值引用的实现可以移出大型成员。\par






















































