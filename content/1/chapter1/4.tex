最后,不能移动用\textit{const}声明的对象。因为任何优化实现都要求可以修改传递的实参,如果不允许修改,就不能窃取。\par

使用\textit{push\_back()}的重载:\par

\begin{lstlisting}[caption={}]
template<typename T>
class vector {
	public:
	...
	// insert a copy of elem:
	void push_back (const T& elem);
	// insert elem when the value of elem is no longer needed:
	void push_back (T&& elem);
	...
};
\end{lstlisting}

对\textit{const}对象使用的函数是具有\textit{const\&}形参的\textit{push\_back()}重载:\par

\begin{lstlisting}[caption={}]
std::vector<std::string> coll;
const std::string s{"data"};
...
coll.push_back(std::move(s)); // OK, calls push_back(const std::string&)
\end{lstlisting}

\textit{const}对象的\textit{std::move()}没起作用。\par

原则上,可以通过声明带有\textit{const}右值引用的函数进行重载,但在语义上没有意义。同样,\textit{const}左值引用方式会作为处理这种情况的备选。\par

\hspace*{\fill} \par %插入空行
\textbf{1.4.1 const返回值}

\textit{const}禁用移动语义对声明返回类型也有影响。\textit{const}返回值不能移动。\par

因此,从C++11开始,用\textit{const}返回值就不再是好的方式了(正如过去的一些风格指南所推荐的那样)。例如:\par

\begin{lstlisting}[caption={}]
const std::string getValue();

std::vector<std::string> coll;
...
coll.push_back(getValue()); // copies (because the return value is const)
\end{lstlisting}

当按值返回时,不要将整个返回值声明为\textit{const}。仅在声明部分返回类型时使用\textit{const}(例如:返回的引用或指针所指向的对象):\par

\begin{lstlisting}[caption={}]
const std::string getValue(); // BAD: disables move semantics for return values
const std::string& getRef(); // OK
const std::string* getPtr(); // OK
\end{lstlisting}





