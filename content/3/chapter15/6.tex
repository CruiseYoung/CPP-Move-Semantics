C++标准库有两个特殊的类型,代表正在运行的线程或用于同步。其中一些既不能复制也不能移动(例如,原子类型或条件变量)。然而,其中一些是只移动类型。让我们从移动语义的角度来看一些细节。\par

\hspace*{\fill} \par %插入空行
\textbf{15.6.1 std::thread<>和std::jthread<>}

表示运行线程的对象(std::thread或C++20 std::jthread)可以传递,但不能复制。可以把它们放在容器里。\par

考虑以下例子:\par

{\color{red}{lib/thread.cpp}}\par

\begin{lstlisting}[caption={}]
#include <iostream>
#include <thread>
#include <vector>

void doThis(const std::string& arg) {
	std::cout << "doThis(): " << arg << '\n';
}
void doThat(const std::string& arg) {
	std::cout << "doThat(): " << arg << '\n';
}

int main()
{
	std::vector<std::thread> threads; // better std::jthread since C++20
	
	// start a couple of threads:
	std::string someArg{"Black Lives Matter"};
	threads.push_back(std::thread{doThis, someArg});
	threads.push_back(std::thread{doThat, std::move(someArg)});
	...
	
	// wait for all threads to end:
	for (auto& t : threads) {
		t.join();
	}
}
\end{lstlisting}

启动两个线程,并将它们放入所有运行线程的集合中:\par

\begin{lstlisting}[caption={}]
std::vector<std::thread> threads; // better std::jthread since C++20

std::string someArg{"black lives matter"};
threads.push_back(std::thread{doThis, someArg});
threads.push_back(std::thread{doThat, std::move(someArg)});
\end{lstlisting}

通过声明std::thread类型的对象(从C++20开始,最好使用std::jthread)来启动一个线程,并将临时对象移动到线程中。在可调用对象(函数, Lambda, 函数对象)的参数后面,可以传递额外的参数,这些参数在线程启动时传递给可调用对象。默认情况下,线程类的构造函数会复制这些参数。使用\textit{std::move()}切换到移动语义。因此,\textit{doThis()}获得传递给\textit{someArg}字符串的副本,而\textit{doThat()}获得该字符串的移动值。\par

最后,等待所有线程的结束(使用std::thread时,这是避免段错误的必要条件):\par

\begin{lstlisting}[caption={}]
// wait for all threads to end:
for (auto& t : threads) {
	t.join();
}
\end{lstlisting}

\hspace*{\fill} \par %插入空行
\textbf{15.6.2 Future, Promise和打包任务}

对于一些用于同步两个线程之间(返回)值交换的helper类型,还使用了只移动类型(future、promise和打包任务都是只移动类型)。\par

考虑以下例子:\par

{\color{red}{lib/future.cpp}}\par

\begin{lstlisting}[caption={}]
#include <iostream>
#include <string>
#include <vector>
#include <thread>
#include <future>

void getValue(std::promise<std::string> p)
{
	try {
		std::string ret{"vote"};
		...
		// store result:
		p.set_value_at_thread_exit(std::move(ret));
	}
	catch(...) {
		// store exception:
		p.set_exception_at_thread_exit(std::current_exception());
	}
}
int main()
{
	std::vector<std::future<std::string>> results;
	
	// create promise and future to deal with outcome of the thread started:
	std::promise<std::string> p;
	std::future<std::string> f{p.get_future()};
	results.push_back(std::move(f));
	
	// start thread and move the promise to it:
	std::thread t{getValue, move(p)};
	t.detach(); // would not be necessary for std::jthread
	...
	
	// wait for all threads to end:
	for (auto& fut : results) {
		std::cout << fut.get() << '\n';
	}
}
\end{lstlisting}

main()中,首先创建了promise,即能够将结果(值或异常)发送给能够读取它的关联future:\par

\begin{lstlisting}[caption={}]
std::promise<std::string> p;
std::future<std::string> f{p.get_future()};
\end{lstlisting}

future移动到future的集合中:\par

\begin{lstlisting}[caption={}]
results.push_back(std::move(f));
\end{lstlisting}

也可以将\textit{p.get\_future()}的结果直接传递给\textit{push\_back()}:\par

\begin{lstlisting}[caption={}]
results.push_back(p.get_future());
\end{lstlisting}

promise移动到已启动的线程:\par

\begin{lstlisting}[caption={}]
std::thread t{getValue, move(p)};
\end{lstlisting}

线程将作为\textit{getValue()}的参数,\textit{getValue()}按值接受传递的promise:\par

\begin{lstlisting}[caption={}]
void getValue(std::promise<std::string> p)
{
	...
}
\end{lstlisting}

通过这种方式,getValue()接收promise的所有权,并在线程结束时销毁通信机制的源.\par

getValue()也可以通过引用的方式获取promise,因为启动的线程保存着已移动的promise值,直到结束为止。\par














