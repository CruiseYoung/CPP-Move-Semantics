只移动类型和对象在声明或使用时有一些共同点。\par

\hspace*{\fill} \par %插入空行
\textbf{13.1.1 声明只移动类型}

只移动类型禁用了复制。通常,复制构造函数和复制赋值操作符会删除,而移动构造函数和移动赋值操作符会默认或实现:\par

例如:\par

\begin{lstlisting}[caption={}]
class MoveOnly {
public:
	// constructors:
	MoveOnly();
	...
	// copying disabled:
	MoveOnly(const MoveOnly&) = delete;
	MoveOnly& operator= (const MoveOnly&) = delete;
	// moving enabled:
	MoveOnly(MoveOnly&&) noexcept;
	MoveOnly& operator= (MoveOnly&&) noexcept;
};
\end{lstlisting}

按照规则,声明移动特殊成员函数就足够了(因为声明特殊移动成员将复制成员标记为已删除)。但是,显式地将复制特殊成员函数标记为=delete会使意图更加明确。\par

\hspace*{\fill} \par %插入空行
\textbf{13.1.2 使用只移动类型}

通过上面的声明,可以创建和移动对象,但不能复制。例如:\par

\begin{lstlisting}[caption={}]
std::vector<MoveOnly> coll;
...
coll.push_back(MoveOnly{}); // OK, creates a temporary object, which is moved into coll
...
MoveOnly mo;
coll.push_back(mo); // ERROR: can’t copy mo into coll
coll.push_back(std::move(mo)); // OK, moves mo into coll
\end{lstlisting}

要将只移动元素的值移出容器,需要使用std::move()作为元素的引用。例如:\par

\begin{lstlisting}[caption={}]
mo = std::move(coll[0]); // move assign first element (still there with moved-from state)
\end{lstlisting}

但请记住,在这个调用之后,元素仍然在容器中,状态为已移动。\par

在循环中,移除所有元素也是可能的:\par

\begin{lstlisting}[caption={}]
for (auto& elem : coll) { // note: non-const reference
	coll2.push_back(std::move(elem)); // move element to coll2
}
\end{lstlisting}

同样的,元素仍然在容器中,状态为已移动。\par

对于只移动类型,有两个操作是不可能的:\par

\begin{itemize}
	\item 不能使用std::initializer\_lists,因为它们通常是按值传递的,这需要复制元素:
	\begin{lstlisting}[caption={}]
	std::vector<MoveOnly> coll{ MoveOnly{}, ... }; // ERROR
	\end{lstlisting}
	\item 只能通过引用迭代一个容器中所有只允许移动的元素:
	\begin{lstlisting}[caption={}]
	std::vector<MoveOnly> coll;
	...
	for (const auto& elem : coll) { // OK
		...
	}
	...
	for (auto elem : coll) { // ERROR: can’t copy move-only elements
		...
	}
	\end{lstlisting}
\end{itemize}

请参阅lib/moveonly.cpp来获得一个完整的程序,其中包含本节中的所有示例语句。\par

\hspace*{\fill} \par %插入空行
\textbf{13.1.3 将只移动对象作为参数传递}

如果使用了移动语义,可以通过值传递和返回只移动的对象:\par

\begin{lstlisting}[caption={}]
void sink(MoveOnly arg); // sink() takes ownership of the passed argument

sink(MoveOnly{}); // OK, moves temporary objects to arg
MoveOnly mo;
sink(mo); // ERROR: can’t copy mo to arg
sink(std::move(mo)); // OK, moves mo to arg because passed by value
\end{lstlisting}

从语义上讲,在这里将相关资源的所有权传递给函数。但请注意,只有当参数按值接受时才会出现这种情况。\par

sink()函数也可以通过(rvalue或通用)引用声明为只接受移动对象。这种情况下,仍然需要通过std::move()传递lvalue。但是,不知道传递的资源的所有权是否由sink()获取。\par

\begin{lstlisting}[caption={}]
void sink(MoveOnly&& arg); // sink() might take ownership of the passed argument

MoveOnly mo;
sink(mo); // ERROR: can’t pass lvalue mo to arg
sink(std::move(mo)); // OK, might move mo to something inside sink()
\end{lstlisting}

Scott Meyers和Herb Sutter (C++社区的两位主要作者)讨论了应该如何声明只移动类型的接收器函数。Herb的立场是通过值来取参数,Scott的立场是通过rvalue引用来取参数。\par

据我所知,他们后来一致认为最好采用rvalue引用的方法。然而,真正的答案不重要。std::move()的通常规则也应该适用于这里:如果用std::move()传递一个只移动的对象,可能会丢失它的值,也可能不会。如果放弃所有权很重要(希望确保文件已关闭、线程已停止或相关的资源已释放),请在调用后使用相应语句直接明确。例如:\par

\begin{lstlisting}[caption={}]
MoveOnly mo;

foo(std::move(mo)); // might move ownership
// ensure mo’s resource is longer acquired/owned/open/running:
mo.close(); // or mo.reset() or mo.release() or so
\end{lstlisting}

只移动对象通常有这样的函数,但名称不同(例如,在C++标准库中,称为流的close(),线程的join(),或唯一指针的reset())。这些函数通常将对象带入默认构造状态。\par

\hspace*{\fill} \par %插入空行
\textbf{13.1.4 按值返回只移动的对象}

还可以实现按值返回(新的)只移动对象的源函数,这意味着将所有权传递给函数的调用者。\par

如果以这种方式返回一个局部对象,就会自动使用移动语义:\par

\begin{lstlisting}[caption={}]
MoveOnly source()
{
	MoveOnly mo;
	...
	return mo; // moves mo to the caller
}

MoveOnly m{source()}; // takes ownership of the associated value/resource
\end{lstlisting}

当有非本地数据时,才可能需要在return语句中使用std::move()(例如,在成员函数中移出成员的值)。\par

\hspace*{\fill} \par %插入空行
\textbf{13.1.5 只移动对象的移动状态}

只移动对象的已移动状态通常不再拥有自己的资源。这是一个已定义的状态,该类型的用户应该能够对其进行双重检查。有时,移动操作只是交换内部数据,以便移动赋值将另一个资源分配给已移动对象(例如,流类这样做)。\par

C++标准库使用不同的方式和名称来检查不再拥有资源的“已移动”状态。例如,正校验(是否仍拥有资源?)可能如下所示:\par

\begin{itemize}
	\item if(s.is\_open())是文件流
	\item if(up)对于unique指针
	\item if(t.joinable())对于thread对象
	\item if(f.valid())对于std::future对象
\end{itemize}














