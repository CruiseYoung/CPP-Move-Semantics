C++标准库提供了一些移动元素的算法。这些算法是:\par

\begin{itemize}
	\item std::move(),将元素移动到另一个范围或在同一范围内向后移动(不要将此算法与std::move()标记不再需要其值的对象混淆)\par
	指定目标范围的开头,元素将从源范围的开头移动到末尾。
	\item std::move\_backward(),将元素移动到另一个范围或在同一范围内向前移动\par
	指定目标范围的结束,元素将从源范围的结束移动到开始。
\end{itemize}

这些算法是使用移动语义的std::copy()和std::copy\_backward()算法的对等操作。是的,std::move()还有另一个重载,它接受多个形参(三个迭代器(从C++17起),还有一个可选的并行执行策略)。\par

这些算法的效果是在迭代每个元素时调用std::move(elem)对目标范围进行移动赋值。\par

考虑以下例子:\par

{\color{red}{lib/movealgo.cpp}}\par

\begin{lstlisting}[caption={}]
#include <iostream>
#include <string>
#include <vector>
#include <list>
#include <algorithm>

template<typename T>
void print(const std::string& name, const T& coll)
{
	std::cout << name << " (" << coll.size() << " elems): ";
	for (const auto& elem : coll) {
		std::cout << " '" << elem << "'";
	}
	std::cout << "\n";
}

int main(int argc, char** argv)
{
	std::list<std::string> coll1 { "love", "is", "all", "you", "need" };
	std::vector<std::string> coll2;
	
	// ensure coll2 has enough elements to overwrite their values:
	coll2.resize(coll1.size());
	
	// print out size and values:
	print("coll1", coll1);
	print("coll2", coll2);
	// move assign the values from coll1 to coll2
	// - not changing any size
	std::move(coll1.begin(), coll1.end(), // source range
	coll2.begin()); // destination range
	
	// print out size and values:
	print("coll1", coll1);
	print("coll2", coll2);
	
	// move assign the first three values inside coll2 to the end
	// - not changing any size
	std::move_backward(coll2.begin(), coll2.begin()+3, // source range
	coll2.end()); // destination range
	
	// print out size and values:
	print("coll1", coll1);
	print("coll2", coll2);
}
\end{lstlisting}

当调用std::move()算法时,将源容器的所有值赋给目标容器:\par

\begin{lstlisting}[caption={}]
// move assign the values from coll1 to coll2
// - not changing any size
std::move(coll1.begin(), coll1.end(), // source range
		  coll2.begin()); // destination range
\end{lstlisting}

与覆盖算法一样,目标容器必须已经有足够的元素(否则就会有未定义的行为)。元素的数量不会改变(无论是在源范围还是在目标范围)。但是,源范围的元素处于“已移动”状态。因此,在这个调用之后,我们不知道源范围内字符串的值(除非为已移动对象指定了行为,比如只移动类型)。\par

当调用本例中的std::move\_backward()算法时,将前三个元素赋值到同一个集合的末尾:\par

\begin{lstlisting}[caption={}]
// move assign the first three values inside coll2 to the end
// - not changing any size
std::move_backward(coll2.begin(), coll2.begin()+3, // source range
				   coll2.end()); // destination range
\end{lstlisting}

同样,如果没有另一个值移走,元素的状态将会是已移动。因此,不再知道前两个元素的值(第三个元素的值被第一个元素的移动赋值覆盖)。在这通调用之后,我们只知道coll2的最后三个元素是love, is和all。\par

因此,整个程序的输出是这样的(?表示不确定的值):\par

\begin{tcolorbox}[colback=white,colframe=black]
	coll1 (5 elems): 'love' 'is' 'all' 'you' 'need' \\
	coll2 (5 elems): '' '' '' '' '' \\
	coll1 (5 elems): '?' '?' '?' '?' '?' \\
	coll2 (5 elems): 'love' 'is' 'all' 'you' 'need' \\
	coll1 (5 elems): '?' '?' '?' '?' '?' \\
	coll2 (5 elems): '?' '?' 'love' 'is' 'all'
\end{tcolorbox}

从字符串移动通常是空的,但这不能保证。实践中,我甚至在一个平台上发现了这样的输出:\par

\begin{tcolorbox}[colback=white,colframe=black]
	coll1 (5 elems): 'love' 'is' 'all' 'you' 'need' \\
	coll2 (5 elems): '' '' '' '' '' \\
	coll1 (5 elems): '' '' '' '' '' \\
	coll2 (5 elems): 'love' 'is' 'all' 'you' 'need' \\
	coll1 (5 elems): '' '' '' '' '' \\
	coll2 (5 elems): 'need' 'you' 'love' 'is' 'all'
\end{tcolorbox}


所以,与往常一样:在使用已移动元素时要小心。\par
































