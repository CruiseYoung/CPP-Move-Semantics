根据设计,C++算法使用迭代器来处理容器和范围的元素。然而,就像指针操作数组一样,迭代器只能读取和写入值,不能插入或删除元素。因此,“删除”算法并不是真正删除元素,只是将所有未移除元素的值移动到已处理范围的前面。\par

例如,给定以下整数序列:\par

\begin{tcolorbox}[colback=white,colframe=black]
1 2 3 4 5 4 3 2 1
\end{tcolorbox}	

调用\textit{std::remove()}算法来删除值为2的所有元素,修改后的序列:\par

\begin{tcolorbox}[colback=white,colframe=black]
1 3 4 5 4 3 1 2 1
\end{tcolorbox}	

所有不是2的元素都移到前面,作为新的结束(“last”元素后面的位置),返回2的位置。\par

可能的话,这些算法会移动,会以已移状态保留元素。本例中,如果元素是字符串,那么最后2个元素将保持不变,但最后1个元素将向前移动到2之前,以便最后一个元素处于已移动状态。\par

因此,这些算法也可以将元素保留为已移动状态。事实上,以下算法可以创建已移动状态:\par

\begin{itemize}
	\item \textit{std::remove()}和\textit{std::remove\_if()}
	\item \textit{std::unique()}
\end{itemize}

看一个完整的例子,可以看到元素是否为已移动状态:\par

{\color{red}{lib/email.hpp}}\par

\begin{lstlisting}[caption={}]
#include <iostream>
#include <cassert>
#include <string>

// class for email addresses
// - asserts that each email address has a @
// - except when in a moved-from state
class Email {
	private:
	std::string value; // email address
	bool movedFrom{false}; // special moved-from state
	public:
	Email(const std::string& val)
	: value{val} {
		assert(value.find('@') != std::string::npos);
	}
	Email(const char* val) // enable implicit conversions for string literals
	: Email{std::string(val)} {
	}
	std::string getValue() const {
		assert(!movedFrom); // or throw
		return value;
	}
	...
	// implement move operations to signal a moved-from state:
	Email(Email&& e) noexcept
	: value{std::move(e.value)}, movedFrom{e.movedFrom} {
		e.movedFrom = true;
	}
	Email& operator=(Email&& e) noexcept {
		value = std::move(e.value);
		movedFrom = e.movedFrom;
		e.movedFrom = true;
		return *this;
	}
	// enable copying:
	Email(const Email&) = default;
	Email& operator=(const Email&) = default;
	
	// print out the current state (even if it is a moved-from state):
	friend std::ostream& operator<< (std::ostream& strm, const Email& e) {
		return strm << (e.movedFrom ? "MOVED-FROM" : e.value);
	}
};
\end{lstlisting}

通过实现移动构造函数和移动赋值操作符,设置了一个成员\textit{movedFrom},并在输出操作符中求值。\par

现在让我们将该类中的一些元素放入vector中,并使用算法\textit{std::remove\_if()}删除一些元素。本例中,我们删除所有以".de"结尾的电子邮件地址:\par

{\color{red}{lib/removeif.cpp}}\par

\begin{lstlisting}[caption={}]
#include "email.hpp"
#include <iostream>
#include <string>
#include <vector>
#include <algorithm>
int main()
{
	std::vector<Email> coll{ "tom@domain.de", "jill@company.com",
		"sarah@domain.de", "hana@company.com" };
	
	// remove all email addresses ending with ".de":
	auto newEnd = std::remove_if(coll.begin(), coll.end(),
	[] (const Email& e) {
		auto&& val = e.getValue();
		return val.size() > 2 &&
		val.substr(val.size()-3) == ".de";
	});

	// print elements up to the new end:
	std::cout << "remaining elements:\n";
	for (auto pos = coll.begin(); pos != newEnd; ++pos) {
		std::cout << " '" << *pos << "'\n";
	}

	// print all elements in the container:
	std::cout << "all elements:\n";
	for (const auto& elem : coll) {
		std::cout << " '" << elem << "'\n";
	}
}
\end{lstlisting}

程序输出如下:\par

\begin{tcolorbox}[colback=white,colframe=black]
remaining elements: \\
"jill@company.com" \\
"hana@company.com" \\
all elements: \\
"jill@company.com" \\
"hana@company.com" \\
"sarah@domain.de" \\
"MOVED-FROM"
\end{tcolorbox}	

“删除”以“.de”结尾的元素之后,新的结束位置是第二个元素后面的位置。但在容器中,第三个元素没有移动,第四个元素处于已移动状态,因为它移动赋值给了第二个元素(之前移动到第一个元素)。\par

使用这些已移动对象时要小心,原因在关于已移动状态的章节中已经讨论过了。\par












