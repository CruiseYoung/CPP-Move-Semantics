通过使用移动迭代器(在C++11中引入),甚至可以在其他算法中使用移动语义,也可以在任何输入范围中使用移动语义(例如,在构造函数中)。\par

但是,使用这些迭代器时要小心。在迭代容器或范围的元素时,每次对元素的访问都使用\textit{std::move()}。这可能会更快,会使元素处于有效但未定义的状态,所以一个元素不应该使用两次\textit{std::move()}。\par

\hspace*{\fill} \par %插入空行
\textbf{14.3.1 移动算法中的迭代器}

算法中使用移动迭代器,通常只有在算法保证每个元素只使用一次时才有意义。因此,该算法应该:\par

\begin{itemize}
	\item 要求源的输入迭代器类别和目标的输出迭代器类别
	\item 或者保证每个元素只使用一次(例如,为\textit{std::for\_each()}算法指定)
\end{itemize}

对于具有可调用对象的算法,允许规范提供详细的功能,元素通过\textit{std::move()}传递给可调用对象。在可调用对象内部,可以决定如何处理:\par

\begin{itemize}
	\item 按值接受参数,总是移动/窃取值或资源
	\item 通过右值/通用引用参数来决定移动/窃取哪个值/资源
\end{itemize}

例如:\par

{\color{red}{lib/foreachmove.cpp}}|\par

\begin{lstlisting}[caption={}]
#include <iostream>
#include <string>
#include <vector>
#include <algorithm>
template<typename T>
void print(const std::string& name, const T& coll)
{
	std::cout << name << " (" << coll.size() << " elems): ";
	for (const auto& elem : coll) {
		std::cout << " \"" << elem << "\"";
	}
	std::cout << "\n";
}
void process(std::string s) // gets moved value from rvalues
{
	std::cout << "- process(" << s << ")\n";
	...
}
int main()
{
	std::vector<std::string> coll{"don't", "vote", "for", "liars"};
	print("coll", coll);
	// move away only the elements processed:
	std::for_each(std::make_move_iterator(coll.begin()),
	std::make_move_iterator(coll.end()),
	[] (auto&& elem) {
		if (elem.size() != 4) {
			process(std::move(elem));
		}
	});
	print("coll", coll);
}
\end{lstlisting}

代码中,将所有大小不为4的元素移动到helper函数\textit{process()}中:\par

\begin{lstlisting}[caption={}]
// move away only the elements processed:
std::for_each(std::make_move_iterator(coll.begin()),
std::make_move_iterator(coll.end()),
[] (auto&& elem) {
	if (elem.size() != 4) {
		process(std::move(elem));
	}
});
\end{lstlisting}

为此,将标记的元素(移动迭代器标记为\textit{std::move()})通过通用引用传递给\textit{process()},并将其与\textit{std::move()}一起传递给\textit{process()}。因为\textit{process()}按值接受参数,所以值实际上已经移走。\par

因此,所有大小不为4的元素都从容器\textit{coll}中的对象中移出。该程序的输出如下(?表示未知值):\par

\begin{tcolorbox}[colback=white,colframe=black]
coll (4 elems): "don't" "vote" "for" "liars" \\
- process(don't) \\
- process(for) \\
- process(liars) \\
coll (4 elems): "?" "vote" "?" "?"
\end{tcolorbox}

最后一行是这样的:\par

\begin{tcolorbox}[colback=white,colframe=black]
coll (4 elems): "" "vote" "" ""
\end{tcolorbox}

这里使用了辅助函数\textit{std::make\_move\_iterator()},这样在声明迭代器时就不必指定元素类型了。从C++17开始,类模板实参推导(CTAD)允许直接声明类型\textit{std::move\_iterator},而不需要指定元素类型:\par

\begin{lstlisting}[caption={}]
std::for_each(std::move_iterator{coll.begin()},
std::move_iterator{coll.end()},
[] (auto&& elem) {
	if (elem.size() != 4) {
		process(std::move(elem));
	}
});
\end{lstlisting}

\hspace*{\fill} \par %插入空行
\textbf{14.3.2 在构造函数和成员函数中移动迭代器}

也可以在只读取一次元素的算法中使用移动迭代器,场景可能是将源容器的元素移动到另一个容器(相同或不同类型的)。例如:\par

{\color{red}{lib/moveitor.cpp}}\par

\begin{lstlisting}[caption={}]
#include <iostream>
#include <string>
#include <list>
#include <vector>

template<typename T>
void print(const std::string& name, const T& coll)
{
	std::cout << name << " (" << coll.size() << " elems): ";
	for (const auto& elem : coll) {
		std::cout << " \"" << elem << "\"";
	}
	std::cout << "\n";
}

int main()
{
	std::list<std::string> src{"don't", "vote", "for", "liars"};
	// move all elements from the list to the vector:
	std::vector<std::string> vec{std::make_move_iterator(src.begin()),
		std::make_move_iterator(src.end())};
	print("src", src);
	print("vec", vec);
}
\end{lstlisting}

该程序有以下输出:(?表示未知值):\par

\begin{tcolorbox}[colback=white,colframe=black]
src (4 elems): "?" "?" "?" "?" \\
vec (4 elems): "don't" "vote" "for" "liars"
\end{tcolorbox}

请再次注意,源容器中的元素数量没有改变。所有元素会移动到初始化的新容器中。因此,源范围中的元素随后处于已移动状态,所以不知道它们的值。\par






































