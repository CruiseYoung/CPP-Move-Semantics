\begin{flushright}
	\zihao{0} 前言
\end{flushright}

C++11添加的Move语义已经成为现代C++的标志,也使语言变得复杂,即使经验丰富的开发者仍在需要仔细处理Move语义的细节。因为这个原因,一些编程书籍甚至不推荐对非常简单的类使用Move语义。所以,详细的解释C++ Move语义就变得刻不容缓。\par

目前为止,每当教授Move语义时,我都会说,“必须有人写一本关于Move的书,”通常的回答是:“那必须的的!你来写吧!”。现在,我终于做到了。\par

和往常一样,写一本关于C++的书时,我都会对要介绍的知识、要阐明的情况和要描述的结果感到惊讶。是时候出一本关于Move语义的书了,要覆盖了从C++11到C++20的所有版本。我在这个过程中学到了很多,我相信你也一样。 \par

\hspace*{\fill} \par %插入空行
\textbf{本书就是个实验}

本书完成了实验的两个方面:\par
\begin{itemize}
	\item 写一本有深度的书,会涉及一个复杂的核心语言特性,但没有核心语言专家的直接帮助。不过,我可以问问题,我也这么做了。
	\item 我自己在Leanpub上出版这本书,并按需印刷。这本书是逐步完成的,一旦有显著的改进,我将发布一个新版本。
\end{itemize}

好的方面是:\par
\begin{itemize}
	\item 您可以从经验丰富的编程人员那里了解语言特性——他们体会到某个特性可能造成的痛苦,并提出相关问题,以便能够激励和解释设计,以及其在实践中编程的结果。
	\item 当我还在写作时,您可以阅读我的Move语义经验,并从中获益。
	\item 这本书和所有的读者都能从您的早期反馈中受益。
\end{itemize}

这意味着你也是实验的一部分。所以,请帮助我:对本书的缺陷、错误、未解释清楚的功能或差距给予反馈,这样我们都能从这些改进中受益。\par

\hspace*{\fill} \par %插入空行
\textbf{致谢}

首先,我要感谢C++社区,是你们使这本书成为可能。令人难以置信的Move语义的功能的设计,有用的反馈,和他们的好奇是语言进化的基础。特别感谢那些告诉我和解释的所有问题的人们,以及给我的反馈,谢谢你们。\par

我要特别感谢每一个为这本书或审阅草稿并提供有价值的反馈和澄清的人。这些评论大大提高了这本书的质量,再次证明了好东西需要许多“聪明人”的投入。目前为止(这个列表还在增长),感谢你们:Javier Estrada, Howard Hinnant, Klaus Iglberger, Daniel Kr ugler, Marc Mutz, Aleksandr Solovev (alexolut), Peter Sommerlad和Tony Van Eerd。\par

此外,我还要感谢C++社区和C++标准委员会的每一个人。除了所有涉及添加新语言和库功能的工作外,这些专家还花了很多很多的时间和我解释和讨论他们的工作,他们很有耐心,也很有热情。\par

特别感谢LaTeX社区提供的文本系统,感谢Frank Mittelbach解决了我的 \LaTeX\xspace 问题。\par

最后,非常感谢本书校对,Tracey Duffy,她做了大量的工作,把我的“德语英语”翻译成地道的英语。\par

\newpage
















