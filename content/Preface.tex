\begin{flushright}
	\zihao{0} 前言
\end{flushright}

Move semantics, introduced with C++11, has become a hallmark of modern C++ programming. However, it also complicates the language in many ways. Even after several years of support, experienced programmers struggle with all the details of move semantics, style guides still recommend different consequences for programming even of trivial classes, and we still discuss semantic details in the C++ standards committee.\par

Whenever I have taught what I have learned about C++ move semantics so far, I have said, “Somebody has to write a book about all this,” and the usual answer was: “Yes, please do!” So, I finally did.\par

As always when writing a book about C++, I was surprised about the number of aspects to be taught, the situations to be clarified, and the consequences to be described. It really was time to write a book about all aspects of move semantics, covering all C++ versions from C++11 up to C++20. I learned a lot and I am sure you will too. \par

\hspace*{\fill} \par %插入空行
\textbf{An Experiment}

This book is an experiment in two ways:\par
\begin{itemize}
	\item I am writing an in-depth book covering a complex core language feature without the direct help of a core language expert as a co-author. However, I can ask questions and I do.
	\item I am publishing the book myself on Leanpub and for printing on demand. That is, this book is written step by step and I will publish new versions as soon there is a significant improvement that makes the publication of a new version worthwhile.
\end{itemize}

The good thing is:\par
\begin{itemize}
	\item You get the view of the language features from an experienced application programmer—somebody who feels the pain a feature might cause and asks the relevant questions to be able to motivate and explain the	design and its consequences for programming in practice.
	\item You can benefit from my experience with move semantics while I am still writing.
	\item This book and all readers can benefit from your early feedback.
\end{itemize}

This means that you are also part of the experiment. So help me out: give feedback about flaws, errors, features that are not explained well, or gaps, so that we all can benefit from these improvements.\par

\hspace*{\fill} \par %插入空行
\textbf{Acknowledgments}

First of all, I would like to thank you, the C++ community, for making this book possible. The incredible design of all the features of move semantics, the helpful feedback, and their curiosity are the basis for the evolution of a successful language. In particular, thanks for all the issues you told me about and explained and for the feedback you gave.\par

I would especially like to thank everyone who reviewed drafts of this book or corresponding slides and provided valuable feedback and clarification. These reviews increased the quality of the book significantly, again proving that good things need the input of many “wise guys.” Therefore, so far (this list is still growing) huge thanks to Javier Estrada, Howard Hinnant, Klaus Iglberger, Daniel Kr¨ugler, Marc Mutz, Aleksandr Solovev (alexolut), Peter Sommerlad, and Tony Van Eerd.\par

In addition, I would like to thank everyone in the C++ community and on the C++ standards committee.In addition to all the work involved in adding new language and library features, these experts spent many, many hours explaining and discussing their work with me, and they did so with patience and enthusiasm.\par

Special thanks go to the LaTeX community for a great text system and to Frank Mittelbach for solving my \LaTeX\xspace issues (it was almost always my fault).\par

And finally, many thanks go to my proofreader, Tracey Duffy, who has done a tremendous job of converting my “German English” into native English.\par

\newpage
















